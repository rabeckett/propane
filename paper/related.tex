\section{Related Work}
\label{sec:related}

Our work builds on three themes of prior work.

\paragraph*{SDN Programming Languages}
\sysname{} was heavily influenced by recent SDN programming
languages such as Merlin~\cite{foster:merlin}, FatTire~\cite{fattire},
NetKAT~\cite{netkat}, as well as path queries~\cite{queries}.
Each of these languages is oriented around regular expressions, which
describe paths through a network, and predicates, which classify packets.
In particular, FatTire allows programmers to define sets of paths together
with a fault tolerance level (\emph{i.e.,} tolerate 1 or 2 faults)
and the compiler will generate appropriate
OpenFlow rules.  \sysname is more
expressive as it allows users to specify preferences among
paths and it automatically generates distributed
implementations that tolerate
any number of faults.  Because FatTire generates data plane rules up front,
before faults occur, specifying higher levels of fault tolerance comes
at the cost of generating additional rules that tax switch
memory.  In contrast, \sysname uses traditional distributed
control plane mechanisms to react to faults, which do not impose
additional memory cost.
Because of the differences in the underlying technology, the analyses
and compilation algorithms used in \sysname are quite different from
previous work on SDN.  Finally, in addition to using path-based abstractions
for intra-domain routing, \sysname uses them for inter-domain routing as
well, unlike any existing SDN language.

\paragraph*{Configuration Synthesis}
ConfigAssure~\cite{narain:lisa05,narain+:configassure}
is another system designed to
help users define and debug low-level router
configurations.  Inputs to
ConfigAssure include a \emph{configuration database}, which contains a
collection of tuples over constants and configuration variables, and a
\emph{requirement}, which is a set of constraints.  For instance, the
tuple \texttt{hsrp(rexa,rA1,int(1),int(2))} states that interface
\texttt{rA1} on device \texttt{rexa} belongs to the HSRP group defined
by configuration variable \texttt{int(1)} with virtual IP address
defined by configuration variable \texttt{int(2)}.  The constraint
\texttt{hsrp\_subnet([rexa-rA1,rexb-rB1])} states that a database
contains tuples defining HSRP group identifiers and virtual IPs for
interfaces \texttt{rA1} and \texttt{rB1} on devices \texttt{rexa} and
\texttt{rexb}.  The authors use a combination of logic programming and
SAT solving to find concrete values for configuration variables such
as \texttt{int(1)}.  ConfigAssure handles
configuration for a wide range of protocols and many
different concerns.  In contrast, the scope of \sysname is much
narrower.  In return, \sysname offers compact, higher-level
abstractions customized for our domain, such as regular paths, as well
as domain-specific analyses customized to those abstractions, such as
our failure safety analysis.  The implementation technology is also
entirely different, as we define algorithms over automata and graphs
as opposed to using logic programming and SAT-based model-finding.

\paragraph*{Configuration Analysis}  The notion that
configuring network devices is difficult and error-prone is not new.  Past
researchers have
tried to tackle this problem by analyzing existing
firewall configurations~\cite{fang,lumeta,margrave} and
BGP configurations~\cite{feamster+:rcc,feamster:thesis,ipassure,batfish,bagpipe} and reporting errors or
inconsistencies when they are detected.
Our research is complementary to these analysis
efforts.  We hope to eliminate bugs by using higher-level
languages and a ``correct-by-construction''
methodology.  By proposing network administrators write configurations
at a high-level of abstraction, a whole host of low-level errors can be
avoided.

%We also believe the high-level, centralized nature of \sysname policies
%has the benefit of making configurations
%easier to understand and maintain.

